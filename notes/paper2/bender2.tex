\documentclass[iop, usenatbib]{emulateapj}

%\usepackage[varg]{txfonts}
\usepackage{amssymb}
\usepackage{amsmath}
\usepackage{epsfig}
\usepackage{graphics}
\usepackage{amsmath}
\usepackage{color}
%\usepackage{xifthen}

%Debug addition for collaborators
%\usepackage[switch, displaymath, modulo]{lineno}
%\linenumbers
%%\renewcommand\linenumberfont{\color{red}\normalfont\tiny\sffamily}
%\renewcommand\linenumberfont{\normalfont\tiny\sffamily}
%\usepackage{natbib,twoopt}


%Collides with emulateapj
%%%%%%%%%
%\usepackage{siunitx}
% Units
%\DeclareSIUnit{\erg}{erg}
%%%%%%%%%


%Commands
\newcommand{\be}{\begin{equation}}
\newcommand{\ee}{\end{equation}}
%\newcommand{\bea}{\begin{eqnarray}}
%\newcommand{\eea}{\end{eqnarray}}
%\newcommand{\bea}{\begin{align}\begin{split}}
%\newcommand{\eea}{\end{split}\end{align}}
\newcommand{\ud}{\text{d}}

%normal 3-vectors
%\renewcommand{\vec}[1]{\ensuremath{\mathbf{#1}}}
\renewcommand{\vec}[1]{\ensuremath{\boldsymbol{#1}}​}

%four-vectors
\makeatletter
\def\fvec#1{\underline{\sbox\tw@{$#1$}\dp\tw@\z@\box\tw@}}
\makeatother

%higlight color
\newcommand{\red}[1]{\textcolor{red}{#1}}

%general shortcuts
\newcommand{\pd}{\ensuremath{\partial}} %partial derivative
\newcommand{\rg}{\ensuremath{r_{\mathrm{g}}}}
\newcommand{\Req}{\ensuremath{R_{\mathrm{e}}}}
\newcommand{\sch}{Schwarzschild }
\newcommand{\Ca}{\ensuremath{\mathcal{C}}}
\newcommand{\kpw}{\ensuremath{\kappa_{\mathrm{PW}}}}

\newcommand{\rb}{\ensuremath{\bar{r}}}
\renewcommand{\ub}{\ensuremath{\bar{u}}}
\newcommand{\wb}{\ensuremath{\bar{\omega}}}
\newcommand{\Ob}{\ensuremath{\hat{\Omega}}}
\newcommand{\nub}{\ensuremath{\bar{\nu}}}
\newcommand{\zetab}{\ensuremath{\bar{\zeta}}}
\newcommand{\Bb}{\ensuremath{\bar{B}}}
\newcommand{\mub}{\ensuremath{\bar{\mu}}}

\newcommand{\vw}{\ensuremath{v_{\omega}}}
\newcommand{\vz}{\ensuremath{v_{\mathrm{z}}}}

\newcommand{\Msun}{\ensuremath{M_{\odot}}}
\newcommand{\lgamma}{\gamma_{\text{L}}}
\newcommand{\qinv}{\ensuremath{q_{\mathrm{inv}}}}
%%%%%%%%%%%%%%%%%%%%%%


\slugcomment{ }
\shorttitle{Polarization from neutron stars}
\shortauthors{N\"attil\"a \& Pihajoki}

\voffset=-1cm

\begin{document}
\title{Polarized radiation from rapidly rotating oblate neutron stars}

\author{J. N\"attil\"a\altaffilmark{1,2}\thanks{nattila.joonas@gmail.com} \textit{et al.} }
%\author{J. N\"attil\"a\altaffilmark{1,2}\thanks{nattila.joonas@gmail.com}
%  P. Pihajoki\altaffilmark{3}}
%  J. Poutanen\altaffilmark{1,2}}

\affil{}
\altaffiltext{1}{Tuorla Observatory, University of Turku}
\altaffiltext{2}{NORDITA, Stockholm}
%\altaffiltext{3}{University of Helsinki}


\begin{abstract}
Framework of formuale is derived for a propagation of polarized radiation originating from rapidly rotating oblate neutron stars.
\end{abstract}
\keywords{gravitation - methods: numerical --- radiative transfer --- stars: neutron}

\section{Introduction}
Blaa





%________________________________________________________________
\section{Theory}\label{sect:theory}

Stokes vector is a handy tool to characterize polarized emission.
For linearly polarized radiation it is given as $(F_I, F_Q, 0, 0)^{\mathrm{T}}$, where $F_I$ is the total flux and $F_Q = P F_I$.
Here $P$ is the polarization degree of the radiation.
When the polarization angle $\chi$ is known, this vector can be rotated from an original fluid frame into the main basis by 
\be
R(\chi) 
\begin{pmatrix}
F_I \\
F_Q \\
0   \\
0   \\
\end{pmatrix}
=
\begin{pmatrix}
F_I \\
F_Q \cos 2\chi\\
F_Q \sin 2\chi  \\
0   \\
\end{pmatrix},
\ee
with a rotation matrix $R(\chi)$ given as
\be
R(\chi) = 
\begin{pmatrix}
1 & 0 & 0 & 0 \\
0 & \cos 2\chi & -\sin 2\chi & 0 \\
0 & \sin 2\chi & \cos 2\chi & 0 \\
0 & 0 & 0 & 1 \\
\end{pmatrix}.
\ee
Total properties of the radiation can then be observed by summing over all individual Stokes vectors to obtain
\be
\begin{pmatrix}
\sum_i F_{I,i} \\[0.3em]
\sum_i \cos 2\chi_i F_{Q,i} \\[0.3em]
\sum_i \sin 2\chi_i F_{Q,i} \\[0.3em]
0 \\
\end{pmatrix}.
\ee

The degree of polarization is scalar invariant.


Given null geodesic (with a given polarization vector) through a point $(t,r,\theta,\phi)$) is charachterized by $L_z$, $E$, $\Ca$, and $\kpw$.

Hughston 1973
Conformal Killing tensor gives rise to a quadratic first integral for null geodesic orbits. 
In general this is a tensor of a valence two.





%\section{Conclusions}
\section*{Acknowledgments}
JN acknowledges support from the University of Turku Graduate School in Physical and Chemical Sciences.



\clearpage

\bibliographystyle{apj}
\bibliography{allbib}

%\begin{thebibliography}
%\end{thebibliography}


\clearpage
\appendix

\section{First blaa}
Blaa blaa

\end{document}
